%% -!TEX root = AUTthesis.tex
% در این فایل، عنوان پایان‌نامه، مشخصات خود، متن تقدیمی‌، ستایش، سپاس‌گزاری و چکیده پایان‌نامه را به فارسی، وارد کنید.
% توجه داشته باشید که جدول حاوی مشخصات پروژه/پایان‌نامه/رساله و همچنین، مشخصات داخل آن، به طور خودکار، درج می‌شود.
%%%%%%%%%%%%%%%%%%%%%%%%%%%%%%%%%%%%
% دانشکده، آموزشکده و یا پژوهشکده  خود را وارد کنید
\faculty{دانشکده مهندسی کامپیوتر}

% عنوان پایان‌نامه را وارد کنید
\fatitle{بررسی چالش‌های پیاده‌سازی محاسبات لبه و تاثیرات آن  بر فناوری اینترنت اشیا
\\[.75 cm]
 پایان‌نامه}
% نام استاد(ان) راهنما را وارد کنید
\firstsupervisor{دکتر صفابخش}
%\secondsupervisor{استاد راهنمای دوم}
% نام استاد(دان) مشاور را وارد کنید. چنانچه استاد مشاور ندارید، دستور پایین را غیرفعال کنید.
\firstadvisor{دکتر صفابخش}
%\secondadvisor{استاد مشاور دوم}
% نام نویسنده را وارد کنید
\name{محمد }
% نام خانوادگی نویسنده را وارد کنید
\surname{اژدری}
%%%%%%%%%%%%%%%%%%%%%%%%%%%%%%%%%%
\thesisdate{تیر ۹۸}

% چکیده پایان‌نامه را وارد کنید
\fa-abstract{با توجه به توسعه و پیشرفت روزافزون اینترنت اشیا در دنیا و نیاز و بازاری که برای آن بوجود آمده است، حجم توسعه این فناوری بالا رفته و جزو فناوری‌های پرطرفدار و به‌روز در دنیا به شمار می‌آید.
از این رو شرکت‌های زیادی در دنیا به توسعه این تکولونوژی و کسب درآمد از آن می‌پردازند.
طبیعی است که چالش‌های مختلفی سر راه توسعه این فناوری قرار دارد.
یکی از آن‌ها حجم بزرگ داده‌ها و محاسباتی است که قرار است روی آن‌ها انجام شود و خروجی مطلوب تولید شود.
در این مقاله سعی شده است به این چالش و راه‌حل‌های ارائه شده برای آن پرداخته شود.
ابتدا این چالش معرفی و بررسی می‌شود و سپس با یک مثال که بررسی معماری یک شهر هوشمند است راه‌حل ارائه می‌شود. و در نهایت با یک نتیحه‌گیری پایان می‌یابد.}


% کلمات کلیدی پایان‌نامه را وارد کنید
\keywords{اینترنت اشیا، محاسبات لبه، شبکه }



\AUTtitle
%%%%%%%%%%%%%%%%%%%%%%%%%%%%%%%%%%
\vspace*{7cm}
\thispagestyle{empty}
\begin{center}
\includegraphics[height=5cm,width=12cm]{besm}
\end{center}