\chapter{فصل ششم تست}
\section{عنوان صفحه تست}
\subsection{زیرعنوان صفحه تست}
قایقی خواهم ساخت،

خواهم انداخت به آب.

دور خواهم شد از این خاک غریب

که در آن هیچ‌کسی نیست که در بیشه عشق

قهرمانان را بیدار کند.

قایق از تور تهی

و دل از آرزوی مروارید،

هم‌چنان خواهم راند.

نه به آبی‌ها دل خواهم بست

نه به دریا-پریانی که سر از خاک به در می‌آرند

و در آن تابش تنهایی ماهی‌گیران

می‌فشانند فسون از سر گیسوهاشان.

هم‌چنان خواهم راند.

هم‌چنان خواهم خواند:

دور باید شد، دور.

مرد آن شهر اساطیر نداشت.

زن آن شهر به سرشاری یک خوشه انگور نبود.

هیچ آیینه تالاری، سرخوشی‌ها را تکرار نکرد.

چاله آبی حتی، مشعلی را ننمود.

دور باید شد، دور.

شب سرودش را خواند،

نوبت پنجره‌هاست.

هم‌چنان خواهم خواند.

هم‌چنان خواهم راند.

پشت دریاها شهری است

که در آن پنجره‌ها رو به تجلی باز است.

بام‌ها جای کبوترهایی است که به فواره هوش بشری می‌نگرند.

دست هر کودک ده ساله شهر، خانه معرفتی است.

مردم شهر به یک چینه چنان می‌نگرند

که به یک شعله، به یک خواب لطیف.

خاک، موسیقی احساس تو را می‌شنود

و صدای پر مرغان اساطیر می‌آید در باد.

پشت دریاها شهری است

که در آن وسعت خورشید به اندازه چشمان سحرخیزان است.

شاعران وارث آب و خرد و روشنی‌اند.

پشت دریاها شهری است!

قایقی باید ساخت. \cite{qiu2020intelligent}

