\chapter{معرفی و مقدمات \lr{edge computing } }

\section{مقدمه}
با توجه به پیشرفت روزافزون تکلونوژی شبکه و اینترنت اشیا و همچنین نیازهایی که به این حوزه به وجود می‌آید، بازار خیلی بزرگی در دنیا برای این صنعت به وجود آمده است.

از این رو شرکت‌های زیادی به منظور سرویس‌دهی بر پایه اینترنت اشیا به وجود آمده‌اند و در حال پیاده‌سازی برنامه‌ها و دستگاه‌های مختلف، چه در قسمت نرم‌افزار و چه در قسمت سخت‌افزار هستند.

در پیاده‌سازی یک پلتفرم اینترنت اشیا، چالش‌های مختلفی وجود دارد.
در میان این‌ها، مسئله حجم بزرگ داده‌ها و محاسبات سنگینی که روی آن‌ها انجام می‌گیرد، یکی از چالش‌های عمده‌ی این کار است.

که در این تحقیق به این چالش و یکی از روش‌های حل آن که \lr{edge computing } است، می‌پردازیم.


\section{محاسبات زیاد، چرا، کجا و چگونه!؟}\label{sec2}
ابتدا کمی درمورد اینترنت اشیا سخن می‌گوییم. اگر بخواهیم به صورت ساده به معرفی این حوزه بپردازیم میتوانیم بگوییم که ارتباط اجزای سخت‌افزاری مختلف مانند  \lr{gadget}های گوناگون تا کارهای بزرگ مانند نیروگاه‌های برق، از طریق اینترنت با هم.

برای مثال فرض کنید یخچال شما آن‌قدر هوشمند است که اگر تخم‌مرغ تمام شد، یا کم مانده بود متوجه میشود و به صورت آنلاین سفارش میدهد و برایتان تخم‌مرغ می‌خرد.
یا یک مثال بزرگ‌تر که یک نیروگاه برق متوجه میشود که در لحظه چقدر برق نیاز است که تولید کند.

همه این‌ها از کوچک به بزرگ دنیای اینترنت اشیا را تشکیل می‌دهند که در دنیای مدرن امروز،‌ روزبه‌روز با پیشرفت همراه است. و نیازش بیشتر احساس می‌شود.
اما همان‌طور که گفتیم، برای پیاده‌سازی این صنعت چالش‌های مختلفی وجود دارد که حجم بزرگ داده‌ها و محاسبات زیاد از اساسی‌ترین آن‌ها به شمار می‌رود.

حالا چرا داده‌ها زیاد هستند؟ دلیلش کمی بارز است. یک شهر هوشمند را تصور کنید که میخواهد حتی خالی بودن سطل آشغال‌های پارک‌هایش را متوجه بشود که مامورهای شهرداری برای خالی کردن آن‌ها اقدام کنند . چراغ راهنمایی‌هایی هوشمندی میخواهد که از ترافیک کل شهر اطلاع داشته باشند و بر اساس آن‌ها عمل کنند.

همه این‌ها نیازمند پردازش و انتقال اطلاعات زیادی است.
حالا این اطلاعات کجا ذخیره می‌شوند و کجا پردازش؟
قاعدتا یک چراغ راهنما آن‌قدر حافظه برای این اطلاعات ندارد و باید از طریق شبکه اینترنت به فضای ابری پلتفرم اصلی وصل شود و به صورت مرتب با آن اطلاعات رد و بدل کند.
بدیهی است که بسیاری از محاسبات را می‌توانیم در همان فضای ابری انجام دهیم و دیگر نیازی نباشد که خود\lr{gadget}  پردازشی انجام بدهد.

خب تا اینجا همه چیز حل شده است و چالشی احساس نمی‌شود!

\section{چالش اصلی کجاست؟}
در برخی از حالت‌ها و دستگاه‌ها آن‌قدر سرعت پردازش و حتی امنیت و قابل اعتماد بودن آن مطرح هست که در مسئله‌ی اینکه این پردازش‌ها کجا انجام پذیرد مسئله مهمی به شمار می‌رود.
برای مثال یک ماشین هوشمند را نظر بگیرید، که واقعا نیاز دارد کاملا اطلاعاتش سربع و دقیق پرداش شوند. حالا اگر این پردازش‌ها در فضای ابری انجام شوند و از طریق اینترنت انجام شوند، ممکن است خطاهایی رخ بدهد و باعث آسیب‌های جبران ناپذیری بشود.

اینجاست که مفهوم \lr{edge computing} مطرح میشود که بیانگر این است که این محاسبات یا در همان خود دستگاه و یا در جایی که هیچ خطای چه شبکه و چه کمبود منابع رخ بدهد، انجام بشوند.
چه در خود دستگاه و چه در جایی نزدیک آن.

\section{برخی مواردی که \lr{edge computing} اهمیت پیدا میکند.}
\subsection{حمل و نقل هوشمند}
ماشین‌های هوشمند را درواقع میتوان کامپیوترهای خیلی هوشمند متحرک درنظر گرفت.
در این ماشین‌هانمی‌توان پردازش‌ها را در فضای ابری انجام داد، زیرا مسیر رفت و برگشت هم کند است و خطاهای زیادی دارد و یک ماشین نمی‌تواند منتظر پاسخ چیزی باشد تا براساس آن تصمیم بگیرد.
در اینجا همه محاسبات باید در نزدیک ترین جای ممکن به سنسورها انجام شده و در سریع‌ترین زمان ممکن بدون خطای انتقال به آن‌ها برسد.
\subsection{حوزه سلامت و پزشکی}
همان‌طور که قبلا هم اشاره شد، اینترنت اشیا در حوزه‌های بسیاری ورود کرده که پزشکی هم از این قاعده مستثنی نیست.
و اتقافا بازارهای بزرگ اینترنت اشیا حساب می‌شود.

حالا اینجا هم ممکن است محاسبات در \lr{cloud}  انجام شوند و ممکن است در نزدیک‌ترین جا به سنسور.
مثلا در سنسورهایی که داخل دستگاه‌های جراحی قرار دارند و یا بعضی از ساعت‌های هوشمند که اطلاعات سلامت افراد را جمع آوری میکنند، پردازش‌ها در لبه سنسور انجام می‌شوند. و نیازی نیست با فضای ابری اطلاعاتی در لحظه رو و بدل شود.
\subsection{کشاورزی}
با توجه به موقعیت مکانی دوردست مزارع ، که بسیاری از آنها دسترسی محدود به پهنای باند دارند، کشاورزی یک نامزد اصلی برای بهره مندی از \lr{edge computing} است. برخلاف ماهواره های تمام وقت یا اتصالات مایکروویو، \lr{edge computing}  گزینه ای پایدار و مقرون به صرفه را ارائه می دهد. مزارع هوشمند می توانند از اینترنت اشیا برای ردیابی مکان تجهیزات ، دمای محیط و عملکرد تجهیزات استفاده کنند.

\subsection{خرده‌فروشی}
آینده سوپر مارکت شامل \textbf{قفسه های هوشمند} با کمک اینترنت اشیا خواهد بود. با استفاده از سنسورهای \lr{RFID (ID Radio Frequency) } در قفسه ها ، شرکت های خرده فروشی قادر به جمع آوری داده ها هستند. در بعضی موارد ، روبات ها قفسه‌ها را اسکن می کنند تا به دنبال مواردی باشند که نیاز به راه اندازی مجدد دارند. برخی از فروشگاه‌های خرده‌فروشی حتی به میهمانان اجازه می دهند با وسایل خود از فروشگاه خارج شوند و کارت های خود را از طریق یک برنامه موبایل شارژ کنند. این امر باعث می شود زمان انتظار برای رونق گرفتن و یک تجربه خرید بدون درز برای مهمانان کاهش یابد.
\subsection{انرژی}
برای شرکت های نفت و گاز و انرژی، \lr{edge computing} می‌تواند تولید، فرآیندها و امنیت را بهبود بخشد.به وفور در بسیاری از مناطق دور افتاده، شرکت های انرژی برای پردازش اطلاعات به اندازه کافی سریع برای رفع نیازهای خود بدون اتصال به فضای ابری، تلاش می کنند. سنسورهایی را در نظر بگیرید که پمپ های نفتی را در این زمینه رصد می کنند که امروز باید از نزدیک با نظارت انسانی کنترل شوند. پردازش داده های زمان واقعی باعث می شود شرکت های انرژی برای هشدار نسبت به مشکلات این پمپ های نفتی به منظور جلوگیری از نظارت غیر ضروری بر هزینه های سربار و بهبود زمان پاسخ دادن به مسئله هشدار دهند.
