\chapter{جمع‌بندي و نتيجه‌گيري و پیشنهادات}
%%%%%%%%%%%%%%%%%%%%%%%%%%%%%%%%%%%%%%%%%%%
\section{جمع‌بندی}

با توجه به توسعه و پیشرفت روزافزون اینترنت اشیا در دنیا و نیاز و بازاری که برای آن بوجود آمده است، حجم توسعه این فناوری بالا رفته و جزو فناوری‌های پرطرفدار و به‌روز در دنیا به شمار می‌آید.
از این رو شرکت‌های زیادی در دنیا به توسعه این تکولونوژی و کسب درآمد از آن می‌پردازند.
طبیعی است که چالش‌های مختلفی سر راه توسعه این فناوری قرار دارد.
یکی از آن‌ها حجم بزرگ داده‌ها و محاسباتی است که قرار است روی آن‌ها انجام شود و خروجی مطلوب تولید شود.
یکی از روش‌های ارائه شده برای حل این موضوع، محاسبات لبه یا همان  \lr{edge computing} است.

این روش می‌کوشد راه‌هایی برای پیدا کردن محل انجام محاسبات پیدا کند و پیشنهاد می‌کند برای دستگاه‌ها و حس‌گر‌هایی که به داده‌های در لحظه و قابل اطمینان نیاز دارند، این محاسبات در نزدیک‌ترین جای ممکن به دستگاه یا همان لبه انجام شوند، تا از سرعت و امنیت کافی برخوردار باشند.

این عملکرد نیز خود ضعف‌ها و قدرت‌هایی دارد و در عین حال همیشه نزدیک‌ترین جا بهترین جا برای محاسبات نیست و به لحاظ هزینه می‌تواند جای محاسبات عوض شود.

در اینجا مفهوم محاسبات لبه هوشمند یا همان \lr{intelligent edge computing} مطرح می‌شود که با تکلونوژی‌هایی مانند یادگیری ماشین و یادگیری عمیق به پیش‌بینی بهترین مکان برای محاسبات می‌پردازد.

\section{نتیجه‌گیری}
با توجه به بررسی‌های انجام شده و اهمیت بیشتر موضوع در مواقع حساس و همچنین در موارد کلان مانند شهرهای هوشمند و مدیریت انرژی به نظر می‌رسد استفاده از محاسبات لبه‌ی هوشمند مناسب‌تر از محاسبات لبه است.
\section{پیشنهادات}
با توجه به مواردی که بیان شد و بازار بزرگ صنعت اینترنت اشیا، پیشنهاد می‌شود که اولا ادامه تحقیقات برای پیاده‌سازی محاسبات لبه هوشمند در عمل و تست‌های مکرر، انجام گیرد.

همچنین بررسی چالش‌های دیگر اینترنت اشیا مانند: امنیت و سرعت و هزینه و ... می‌تواند کمک شایانی به پیشرفت هر چه بیشتر این صنعت داشته باشد.